\documentclass[tcc-proposta]{texufpel}

\usepackage[utf8]{inputenc} % acentuacao
\usepackage{graphicx} % para inserir figuras
\usepackage[T1]{fontenc}
\usepackage{colortbl}

\hypersetup{
	hidelinks, % Remove coloração e caixas
	unicode=true,   %Permite acentuação no bookmark
	linktoc=all %Habilita link no nome e página do sumário
}

\unidade{Centro de Desenvolvimento Tecnológico}
\curso{Engenharia de Computação}
\nomecurso{Bacharelado em Engenharia de Computação}
\titulocurso{Bacharel em Engenharia de Computação}

\title{Proposta de exploração da plataforma Ethereum como base a aplicações em Fog Computing}

\author{Santos}{Gustavo Fernandes dos}
\advisor[Prof.~Dra.]{Reiser}{Renata Hax Sander}
\coadvisor[Prof.~Dr.]{Pilla}{Maurício}
%\collaborator[Prof.~Dr.]{Aguiar}{Marilton Sanchotene de}

\begin{document}

\renewcommand{\advisorname}{Orientadora}           %descomente caso tenhas orientadora
%\renewcommand{\coadvisorname}{Coorientadora}      %descomente caso tenhas co-orientadora
	
\maketitle 
\sloppy
	
\chapter{Dados de Identificação}
	
\section{Nome do Projeto}
Proposta de exploração da plataforma Ethereum como base a aplicações em Fog Computing
	
\section{Local de Realização}
Universidade Federal de Pelotas
	
\section{Responsável pelo Projeto}
Gustavo Fernandes dos Santos
	
gfdsantos@inf.ufpel.edu.br
	
\section{Professor Orientador}
Prof. Renata Hax Sander Reiser
	
\section{Professor Co-orientador}
Prof. Maurício Pilla
	
\chapter{Sumário Executivo}
% (NO MÁXIMO 1 PÁGINA)
	
O principal problema da Internet das Coisas é o crescente uso intenso da comunicação entre os dispositivos e a nuvem. O modelo de Fog Computing promove trazer a nuvem para próximo aos dispositivos que realizam a coleta de dados. Através de nós, o modelo de Fog Computing cria uma camada de processamento e armazenamento na borda da rede ~\cite{Bonomi2012}. A posição de nós que fazem o intermédio entre as coisas e a nuvem reduz consideravelmente a frequência da comunicação com a nuvem e consequentemente reduz o consumo de banda.
	
Ethereum disponibiliza uma plataforma de computação sobre a arquitetura de blockchain ~\cite{dannen2017introducing}. Visando diferentes abordagens de Fog Computing, Ethereum é uma ferramenta que pode trazer todos os recursos necessários para a implementação do modelo de Fog Computing.
	
Este trabalho tem como foco o estudo da arquitetura de blockchain sobre o armazenamento de dados, a capacidade do Ethereum em realizar computações em uma arquitetura de blockchain e como estas tecnologias podem resolver problemas de Fog Computing.
	
	
\chapter{Histórico e Justificativa}
% (NO MÍNIMO 1 PÁGINA)
	
Internet das Coisas é um termo utilizado para designar um conjunto de equipamentos munidos de sensores capazes de formar uma interface entre o mundo físico e o mundo digital. Dispositivos na Internet das Coisas interagem através da internet, tipicamente respondendo a um servidor central - a nuvem, no qual o acesso a dados coletados através destes dispositivos é garantido sempre que a conexão entre as partes for íntegra ~\cite{minerva2015towards}.
	
A crescente adoção de coisas conectadas a internet exige que alguma infraestrutura dê suporte ao intenso tráfego de dados entre os dispositivos e a nuvem. Fornecer um canal de comunicação cuja latência entre as partes não afete o correto funcionamento do sistema, tampouco afete os demais dispositivos que utilizam o mesmo canal de comunicação, é um desafio ~\cite{rosa2017exehda}. Outra limitação do uso de um servidor central é a possibilidade de falha deste servidor. Uma falha em um servidor central compromete o sistema de forma catastrófica, para tal existem heurísticas para assegurar que o sistema continue a funcionar ~\cite{coulouris2005distributed}.

Assegurar que um contexto seja mantido entre os dispositivos e o processamento e armazenamento de dados sem que o sistema seja comprometido de alguma forma, é um desafio. Para tal, é interessante o estudo de diferentes técnicas para garantir a consistência das informações, bem como a continuidade do sistema. Fog Computing tenta resolver este problema trazendo a computação para as bordas da rede através de nós, chamados de nós-fog ~\cite{Bonomi2012}.
	
Segundo \cite{Bonomi2012}, a abordagem do Fog Computing tem a característica de reduzir o consumo de banda na comunicação entre dispositivos e a nuvem, no qual torna-se muito interessante o seu uso quando dados são coletados nas bordas da rede, por exemplo, carros, aviões, rodovias, etc., quando dispositivos estão espalhados por uma grande área geográfica e quando é necessário agir sobre um conjunto de dados em um tempo menor que o necessário caso esta ação seja executada por um serviço em nuvem. Neste cenário, qualquer dispositivo pode ser um nó-fog.
	
Sistemas baseados em Fog Computing podem ser distribuídos ao longo da rede, promovendo maior resiliência para a entrega do serviço. Visando a definição de uma camada abaixo para promover este modelo de computação distribuída, garantir a consistência dos dados e produzir resultados de processamento válidos, uma arquitetura de blockchain pode ser utilizada.
	
Blockchain é um banco de dados distribuído onde são armazenados registros de dados ~\cite{Juels2016}. A grande diferença entre o modelo de blockchain e o armazenamento de dados centralizado é o ponto de falha. Um servidor de armazenamento de dados possui um único ponto de falha, que é justamente o servidor. Já em uma arquitetura de blockchain, cada nó contém uma cópia exata de todos os blocos, o que previne falhas no sistema, manipulação de dados e tentativas em corromper o sistema ~\cite{Juels2016}.
	
Um sistema de Fog Computing baseado em blockchain pode manter a consistência dos dados e ser tolerante a falhas, pois se um ou mais nós forem comprometidos, a entrega do serviço que o sistema dispõe não será interrompida. Embora a abordagem do uso da arquitetura de blockchain seja interessante para o armazenamento de registros de dados, é necessário que ocorra alguma espécie de computação em cada nó do modelo de Fog Computing, para tal a plataforma do Ethereum surge como opção.
    
Ethereum é uma plataforma descentralizada que executa contratos inteligentes. Chamados de \textit{smart contracts}, os programas que executam na plataforma do Ethereum não podem ser parados, não podem ser censurados, não sofrem com interferências de terceiros e não podem ser fraudados ~\cite{wood2014ethereum}, entretanto um contrato tem um limite de instruções nos quais podem ser executadas, este limite é dado pelo \textit{gas}, uma medida de consumo de energia por instrução. Ethereum possui uma estrutura de blockchain onde todos os programas são armazenados e é o local onde são executados. Diferentemente do Bitcoin, Ethereum provê uma plataforma completa para realização de computações e o consenso da rede é mantido via prova de participação ~\cite{buterin2014next}. 
	
Além de ser uma moeda virtual, onde o seu valor é convertido em unidades de computação virtual, a plataforma do Ethereum disponibiliza um ambiente para o desenvolvimento de aplicações baseadas em blockchain ~\cite{wood2014ethereum}. Estas aplicações podem ser conectadas a rede principal do Ethereum, onde são necessários \textit{tokens} reais do Ethereum, chamados de Ether, para a realização de computações ou, então, podem ser executadas em um ambiente privado customizável.
	
Visto que o modelo de Fog Computing exige que nós-fog sejam postos próximos aos dispositivos que realizam a comunicação entre o mundo real e o digital e, ainda, que estes nós realizam processamento e armazenamento de dados, é interessante que um estudo sobre a viabilidade do uso do Ethereum como base para o modelo de Fog Computing, bem como o desenvolvimento de uma prova de conceito.
	
\chapter{Objetivos e Metas}

O objetivo geral deste trabalho é o desenvolvimento de uma aplicação na plataforma Ethereum com suporte a Fog Computing. Para tal existem objetivos intermediários, como a identificação de aplicações que poderiam se beneficiar dos recursos da plataforma Ethereum e a modelagem de uma estrutura de blockchain onde o equilíbrio entre os recursos necessários para mineração e disponibilidade da aplicação.

Metas deste trabalho envolvem a realização de uma revisão técnica sobre a plataforma Ethereum, a identificação das melhores formas de explorar os recursos disponibilizados, o desenvolvimento de uma prova de conceito que valida as ideias postas neste trabalho e a publicação de artigos para fomentar a discussão sobre arquiteturas em blockchain e suas aplicações.
	
\chapter{Metodologia}
A metodologia deste trabalho consiste nos seguintes tópicos:
	
\begin{itemize}
	\item \textbf{Revisão bibliográfica.} Nesta atividade será feita uma revisão sistemática de trabalhos publicados na área de interesse, um estudo sobre o estado da arte de arquiteturas de blockchain, quais problemas são atualmente resolvidos através desta arquitetura e quais os desafios que existem. Será realizado um estudo técnico sobre a plataforma Ethereum e como esta plataforma pode ser utilizada para resolver problemas de Fog Computing.
	
	\item \textbf{Especificação.} Aqui será realizado a especificação de um problema em Fog Computing que se beneficia em ter como base uma arquitetura de blockchain..
	
	\item \textbf{Modelagem.} Será definido um modelo que representa uma solução de um problema em Fog Computing. O modelo definido será apoiado principalmente, mas não somente, em características da plataforma Ethereum.
	
	\item \textbf{Implementação.} Etapa de prototipação da prova de conceito. 
	
	\item \textbf{Validação.} A etapa de validação consiste em avaliar se a solução desenvolvida é correta.
\end{itemize}
	
\chapter{Plano de Atividades e Cronograma}
% (PARA 1 ANO)

	\begin{table}[h!]
		\centering
		\label{my-label}
		\begin{tabular}{llllllllll}
		 & Abr & Mai & Jun & Jul & Ago & Set & Out & Nov & Dez \\
		A1 & \cellcolor[HTML]{9B9B9B} &  &  &  &  &  &  &  &  \\
		A2 &  & \cellcolor[HTML]{9B9B9B} & \cellcolor[HTML]{9B9B9B} &  &  &  &  &  &  \\
		A3 &  &  & \cellcolor[HTML]{9B9B9B} &  &  &  &  &  &  \\
		A4 &  &  & \cellcolor[HTML]{9B9B9B} & \cellcolor[HTML]{9B9B9B} &  &  &  &  &  \\
		A5 &  &  &  & \cellcolor[HTML]{9B9B9B} & \cellcolor[HTML]{9B9B9B} & \cellcolor[HTML]{9B9B9B} & \cellcolor[HTML]{9B9B9B} & \cellcolor[HTML]{9B9B9B} &  \\
		A6 &  &  &  &  &  &  &  & \cellcolor[HTML]{9B9B9B} & \cellcolor[HTML]{9B9B9B} \\
		A7 &  & \cellcolor[HTML]{9B9B9B} & \cellcolor[HTML]{9B9B9B} & \cellcolor[HTML]{9B9B9B} &  &  &  &  &  \\
		A8 &  &  &  & \cellcolor[HTML]{9B9B9B} &  &  &  &  &  \\
		A9 &  &  &  & \cellcolor[HTML]{9B9B9B} & \cellcolor[HTML]{9B9B9B} & \cellcolor[HTML]{9B9B9B} & \cellcolor[HTML]{9B9B9B} & \cellcolor[HTML]{9B9B9B} &  \\
		A10 &  &  &  &  &  &  &  & \cellcolor[HTML]{9B9B9B} &  \\
		A11 &  &  &  &  &  &  &  &  & \cellcolor[HTML]{9B9B9B}
		\end{tabular}
		\end{table}
	
	A1: Escrita da proposta de TCC.
	
	A2: Revisão bibliográfica.
	
	A3: Especificação.
	
	A4: Modelagem.
	
	A5: Implementação.
	
	A6: Validação.
	
	A7: Escrita da monografia parcial.
	
	A8: Entrega da monografia parcial.
	
	A9: Escrita da monografia.
	
	A10: Entrega da monografia para a banca.
	
	A11: Apresentação para a banca.
	
	
\bibliography{bibliografia}
\bibliographystyle{abnt}
	
\chapter{Assinaturas}
\vspace{2cm}
	
\begin{center}
\rule{8cm}{.3mm}
\medskip
	
	Gustavo Fernandes dos Santos\\
	Proponente
	
\end{center}
	
\vspace{4cm}
	
\begin{center}
\rule{8cm}{.3mm}
\medskip
	
	Renata Hax Sander Reiser\\
	Prof. Orientadora
	
\end{center}
\end{document}
	
